\section{不确定性量化}

例如现代电网控制管理中, 风, 光等可再生能源在时空尺度上往往呈现强烈的波动性, 严重影响系统的稳定性, 需要对此类不确定性的量化与精准的预测.

不确定性量化主要考虑系统参数, 模型和计算这三种不确定因素对系统状态的影响:
\begin{itemize}
  \item 数据的不确定性主要由测量与解读的误差, 变化性与多尺度, 稀疏的数据几部分组成, 减少参数(数据)不确定性, 但是无法完全消除它;
  \item 模型不确定性分为系统的组成之不确定性以及物理, 化学和生物过程的不确定性, 也称之为基础不确定性; 如何选取与组合不同的数学模型, 对现实系统状态进行有效的预测是模型不确定性的目标;
  \item 连续的数学方程离散化势必会引入误差不确定性, 进而影响系统状数值拟合.
\end{itemize}

当前不确定性量化的研究热点是参数不确定性. 实际研究中, 将不确定参数等同于随机变量或随机过程, \(Z(\boldsymbol{x},t)\equiv Z(\boldsymbol{x},t;\omega)\), 即参数不仅在时空间 \((\boldsymbol{x},t)\) 变化, 同时也在概率空间 \(\omega\) 变化. 含有这些随机参数的原确定系统随之变为随机系统, 它的解是原系统(状态)输出的统计信息(概率密度函数或概率分布函数).

现有的参数不确定性量化方法可以分为统计型(如 Monte Carlo)和随机数学型(随机有限元, 统计矩微分方程, 摄动法, PDF/CDF 方法等)两大种:
\begin{itemize}
  \item Monte Carlo 模拟可以考虑高维度随机变量, 简单易用; 但收敛速度慢, 需要大量的输出样本, 计算成本高, 无法提供系统状态的随机变化规律. 改进方面主要在提升收敛速度, 如拉丁超立方采样, 准 Monte Carlo 采样等, 但有额外的限制条件, 适用性降低.
  \item 扰动法是非抽样类方法, 无法考虑过多的不确定参数和系统状态(输入和输出的总维度小于 10).
  \item 算子法包含 Neumann 展开和加权积分, 基于系统控制方程随机算子的方法. 也无法考虑过高的随机维度, 且强烈依赖于控制方程的算子. 适用于稳态问题.
  \item 低阶统计矩方程的求解需要高阶统计矩的信息, 引入了闭包问题, 需要具体问题选用特殊的近似方法解决.
  \item PDF/CDF 方法, 计算系统状态的精细概率密度函数或精细概率分布函数方程.
  \item 广义多项式混沌法是经典多项式混沌方法的泛化, 核心是将随机解表示为随机输入的正交多项式, 本质是将随机空间以扰动的形式变现出来, 具有很好的收敛速度.
\end{itemize}


