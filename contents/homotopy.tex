\section{同伦方法}
\subsection{预备知识}
\subsubsection{光滑映射}

\begin{definition}
如果映射 \(f:D\subset\mathbb{R}^m\to\mathbb{R}^n\) 在定义域 \(D\) 中每一点都具有 \(r\) 阶连续偏导数, 则称 \(f\) 为 \(C^r\) 映射; 如果对任一个正整数 \(r\), 映射 \(f\) 是 \(C^r\) 映射, 则称 \(f\) 是光滑映射.
\end{definition}

\begin{enumerate}
    \item 光滑映射在其定义域内每一点处都可微.
    \item 如果 \(f:X\to Y\), \(g:Y\to Z\) 都是光滑映射, 则复合映射 \(g\circ f:X\to Z\) 也是光滑的.
    \item 任意集合上的恒同映射和常值映射都是光滑映射.
    \item \(\mathbb{R}^m\) 中的任意紧集上的连续映射都可由光滑映射任意逼近.
    \begin{theorem}
    设 \(X\subset\mathbb{R}^m\) 是紧集, \(f: X\to\mathbb{R}^n\) 是连续映射, 则对任意 \(\varepsilon > 0\), 存在光滑映射 \(g: X\to\mathbb{R}^n\), 使得对任意 \(x\in X\), 成立
    \begin{equation*}
        \left\lVert f(x)-g(x)\right\rVert < \varepsilon.
    \end{equation*}
    \end{theorem}
\end{enumerate}

\subsubsection{正则值}

\begin{definition}
设 \(f:D\subset\mathbb{R}^m\to\mathbb{R}^n\) 是光滑映射, 对 \(D\) 中的某一点 \(\boldsymbol{x}_0\), 如果 \(f\) 在 \(\boldsymbol{x}_0\) 处的 Jacobi 矩阵 \(\dfrac{\partial f}{\partial x}\left(\boldsymbol{x}_0\right)\) 行满秩, 则称 \(\boldsymbol{x}_0\) 是映射 \(f\) 的正则点. 若 \(\boldsymbol{x}_0\) 不是映射 \(f\) 的正则点, 即映射 \(f\) 在 \(\boldsymbol{x}_0\) 点处的 Jacobi 矩阵行降秩, 则称 \(\boldsymbol{x}_0\) 是映射 \(f\) 的临界点.
\end{definition}

\begin{definition}
设 \(\boldsymbol{y}_0\in\mathbb{R}^n\), 如果所有 \(\boldsymbol{x}_0\in f^{-1}\left(\boldsymbol{y}_0\right)\) 都是映射 \(f\) 的正则点, 则称 \(\boldsymbol{y}_0\) 为映射的正则值; 如果 \(\boldsymbol{y}_0\) 不是映射的正则值, 亦即存在 \(\boldsymbol{x}_0\in f^{-1}\left(\boldsymbol{y}_0\right)\) 使得 \(\boldsymbol{x}_0\) 是 \(f\) 的临界点, 则称 \(\boldsymbol{y}_0\) 是映射 \(f\) 的临界值. 特别地, 如果 \(\boldsymbol{y}_0\notin f(D)=\left\lbrace f(\boldsymbol{x}): \boldsymbol{x}\in D\right\rbrace\), 即 \(\boldsymbol{y}_0\in\mathbb{R}^n\backslash f(D)\), 则 \(\boldsymbol{y}_0\) 是映射 \(f\) 的正则值.
\end{definition}

临界点的像一定是临界值, 但正则点的项不一定是正则值. 只要 \(f^{-1}\left(\boldsymbol{y}_0\right)\) 中有一个临界点, \(\boldsymbol{y}_0\) 就是临界值, 同时 \(f^{-1}\left(\boldsymbol{y}_0\right)\) 中可能有多个正则点.

如果 \(m=n\), 使得 Jacobi 行列式 \(\dfrac{\partial f}{\partial \boldsymbol{x}}\left(\boldsymbol{x}\right)=0\) 的点 \(\boldsymbol{x}\) 称为 \(f\) 的临界点.

\subsubsection{微分同胚}

\begin{definition}
设 \(X\) 和 \(Y\) 分别是两个欧式空间中的子集, 如果映射 \(f:X\to Y\) 是双射(即一一对应), 且 \(f\) 与 \(f\) 的逆映射 \(f^{-1}\) 都是光滑映射, 则称 \(f\) 是 \(X\) 到 \(Y\) 的一个微分同胚. 如果这样的同胚存在, 则称 \(X\) 与 \(Y\) 是微分同胚的.
\end{definition}

\begin{definition}
设 \(X\) 和 \(Y\) 是某两个欧式空间中的子集, 如果 \(f\) 给出 \(X\) 中点 \(\boldsymbol{x}_0\) 的某个邻域到 \(Y\) 中 \(f\left(\boldsymbol{x}_0\right)\) 的某个邻域的微分同胚, 则称映射 \(f:X\to Y\) 在 \(\boldsymbol{x}_0\) 点处是局部微分同胚. 如果 \(f\) 在 \(X\) 的每一点处是局部微分同胚的, 则称 \(f\) 是 \(X\) 到 \(Y\) 的一个局部微分同胚.
\end{definition}

\begin{theorem}[反函数定理]
设 \(W\) 和 \(V\) 是 \(\mathbb{R}^n\) 中的两个开集, \(f:W\to V\) 是光滑映射. 若 \(f\) 在 \(\boldsymbol{x}_0\in W\) 处的导映射 \(\mathrm{d}f_{\boldsymbol{x}_0}\) 是 \(\mathbb{R}^n\) 到 \(\mathbb{R}^n\) 的同构映射, 则 \(f\) 在 \(\boldsymbol{x}_0\) 处是局部微分同胚.
\end{theorem}

\begin{corollary}
设 \(W\) 和 \(V\) 是 \(\mathbb{R}^n\) 中的开集, \(f:W\to V\) 是光滑映射, 则 \(f\) 在 \(\boldsymbol{x}_0\in W\) 是局部微分同胚的充分必要条件是: \(\mathrm{d}f_{\boldsymbol{x}_0}\) 是同构映射.
\end{corollary}

\subsubsection{Sard 定理}

\begin{definition}
设 \(X\) 是 \(\mathbb{R}^n\) 的一个子集, 如果对任意 \(\boldsymbol{x}\in X\), 存在 \(\boldsymbol{x}\) 在 \(X\) 的一个邻域 \(V\subset X\), 使得 \(V\) 与 \(\mathbb{R}^k\) 的一个开集微分同胚, 则称 \(X\) 是 \(k\) 维光滑流形. 若光滑流形 \(X\) 的子集 \(Y\) 也是光滑流形, 就说 \(Y\) 是 \(X\) 的子流形.
\end{definition}

常见的光滑流形:
\begin{itemize}
    \item \(\mathbb{R}^n\) 是 \(n\) 维光滑流形;
    \item \((0,1)\times\mathbb{R}^n\) 是 \(n+1\) 维光滑流形;
    \item \(\mathbb{R}^n\) 中的单位开球 \(B(1)=\lbrace x\in\mathbb{R}^n: \left\lVert x\right\rVert < 1\rbrace\) 和单位球面 \(S(1)=\lbrace x\in\mathbb{R}^n: \left\lVert x\right\rVert =1\rbrace\) 分别是 \(n\) 维光滑流形和 \(n-1\) 维光滑流形.
\end{itemize}

流形之间光滑映射的任一正则值的逆象是一个光滑流形.

\begin{theorem}[逆象定理]
设 \(X\) 和 \(Y\) 分别是 \(k\) 维的和 \(l\) 维的光滑流形, \(k>l\), \(f:X\to Y\) 是光滑映射, 如果 \(y\in Y\) 是映射 \(f\) 的正则值, 则 \(f^{-1}(y)\) 或者是空集, 或者是 \(X\) 中的 \(k-l\) 维子流形.
\end{theorem}

逆象定理表明, 正则值的逆象有很好的几何结构. 那么 \(Y\) 中有多少点是光滑映射 \(f:X\to Y\) 的正则值?

\begin{theorem}[Sard 定理]
设 \(X\) 和 \(Y\) 是光滑流形, \(f:X\to Y\) 是光滑映射. 记 \(D\) 是 \(f\) 的临界点集, 则 \(f\) 的临界点集 \(f(D)\subset Y\) 在 \(Y\) 中的测度为零.
\end{theorem}
