\section{同伦方法}
\subsection{预备知识}
\subsubsection{光滑映射}

\begin{definition}
  如果映射 \(f:D\subset\mathbb{R}^m\to\mathbb{R}^n\) 在定义域 \(D\) 中每一点都具有 \(r\) 阶连续偏导数, 则称 \(f\) 为 \(C^r\) 映射; 如果对任一个正整数 \(r\), 映射 \(f\) 是 \(C^r\) 映射, 则称 \(f\) 是光滑映射.
\end{definition}

\begin{enumerate}
  \item 光滑映射在其定义域内每一点处都可微.
  \item 如果 \(f:X\to Y\), \(g:Y\to Z\) 都是光滑映射, 则复合映射 \(g\circ f:X\to Z\) 也是光滑的.
  \item 任意集合上的恒同映射和常值映射都是光滑映射.
  \item \(\mathbb{R}^m\) 中的任意紧集上的连续映射都可由光滑映射任意逼近.
    \begin{theorem}
      设 \(X\subset\mathbb{R}^m\) 是紧集, \(f: X\to\mathbb{R}^n\) 是连续映射, 则对任意 \(\varepsilon > 0\), 存在光滑映射 \(g: X\to\mathbb{R}^n\), 使得对任意 \(x\in X\), 成立
      \begin{equation*}
        \left\lVert f(x)-g(x)\right\rVert < \varepsilon.
      \end{equation*}
    \end{theorem}
\end{enumerate}

\subsubsection{正则值}

\begin{definition}
  设 \(f:D\subset\mathbb{R}^m\to\mathbb{R}^n\) 是光滑映射, 对 \(D\) 中的某一点 \(\boldsymbol{x}_0\), 如果 \(f\) 在 \(\boldsymbol{x}_0\) 处的 Jacobi 矩阵 \(\dfrac{\partial f}{\partial x}\left(\boldsymbol{x}_0\right)\) 行满秩, 则称 \(\boldsymbol{x}_0\) 是映射 \(f\) 的正则点. 若 \(\boldsymbol{x}_0\) 不是映射 \(f\) 的正则点, 即映射 \(f\) 在 \(\boldsymbol{x}_0\) 点处的 Jacobi 矩阵行降秩, 则称 \(\boldsymbol{x}_0\) 是映射 \(f\) 的临界点.
\end{definition}

\begin{definition}
  设 \(\boldsymbol{y}_0\in\mathbb{R}^n\), 如果所有 \(\boldsymbol{x}_0\in f^{-1}\left(\boldsymbol{y}_0\right)\) 都是映射 \(f\) 的正则点, 则称 \(\boldsymbol{y}_0\) 为映射的正则值; 如果 \(\boldsymbol{y}_0\) 不是映射的正则值, 亦即存在 \(\boldsymbol{x}_0\in f^{-1}\left(\boldsymbol{y}_0\right)\) 使得 \(\boldsymbol{x}_0\) 是 \(f\) 的临界点, 则称 \(\boldsymbol{y}_0\) 是映射 \(f\) 的临界值. 特别地, 如果 \(\boldsymbol{y}_0\notin f(D)=\left\lbrace f(\boldsymbol{x}): \boldsymbol{x}\in D\right\rbrace\), 即 \(\boldsymbol{y}_0\in\mathbb{R}^n\backslash f(D)\), 则 \(\boldsymbol{y}_0\) 是映射 \(f\) 的正则值.
\end{definition}

临界点的像一定是临界值, 但正则点的项不一定是正则值. 只要 \(f^{-1}\left(\boldsymbol{y}_0\right)\) 中有一个临界点, \(\boldsymbol{y}_0\) 就是临界值, 同时 \(f^{-1}\left(\boldsymbol{y}_0\right)\) 中可能有多个正则点.

如果 \(m=n\), 使得 Jacobi 行列式 \(\dfrac{\partial f}{\partial \boldsymbol{x}}\left(\boldsymbol{x}\right)=0\) 的点 \(\boldsymbol{x}\) 称为 \(f\) 的临界点.

\subsubsection{微分同胚}

\begin{definition}
  设 \(X\) 和 \(Y\) 分别是两个欧式空间中的子集, 如果映射 \(f:X\to Y\) 是双射(即一一对应), 且 \(f\) 与 \(f\) 的逆映射 \(f^{-1}\) 都是光滑映射, 则称 \(f\) 是 \(X\) 到 \(Y\) 的一个微分同胚. 如果这样的同胚存在, 则称 \(X\) 与 \(Y\) 是微分同胚的.
\end{definition}

\begin{definition}
  设 \(X\) 和 \(Y\) 是某两个欧式空间中的子集, 如果 \(f\) 给出 \(X\) 中点 \(\boldsymbol{x}_0\) 的某个邻域到 \(Y\) 中 \(f\left(\boldsymbol{x}_0\right)\) 的某个邻域的微分同胚, 则称映射 \(f:X\to Y\) 在 \(\boldsymbol{x}_0\) 点处是局部微分同胚. 如果 \(f\) 在 \(X\) 的每一点处是局部微分同胚的, 则称 \(f\) 是 \(X\) 到 \(Y\) 的一个局部微分同胚.
\end{definition}

\begin{theorem}[反函数定理]
  设 \(W\) 和 \(V\) 是 \(\mathbb{R}^n\) 中的两个开集, \(f:W\to V\) 是光滑映射. 若 \(f\) 在 \(\boldsymbol{x}_0\in W\) 处的导映射 \(\mathrm{d}f_{\boldsymbol{x}_0}\) 是 \(\mathbb{R}^n\) 到 \(\mathbb{R}^n\) 的同构映射, 则 \(f\) 在 \(\boldsymbol{x}_0\) 处是局部微分同胚.
\end{theorem}

\begin{corollary}
  设 \(W\) 和 \(V\) 是 \(\mathbb{R}^n\) 中的开集, \(f:W\to V\) 是光滑映射, 则 \(f\) 在 \(\boldsymbol{x}_0\in W\) 是局部微分同胚的充分必要条件是: \(\mathrm{d}f_{\boldsymbol{x}_0}\) 是同构映射.
\end{corollary}

\subsubsection{Sard 定理}

\begin{definition}
  设 \(X\) 是 \(\mathbb{R}^n\) 的一个子集, 如果对任意 \(\boldsymbol{x}\in X\), 存在 \(\boldsymbol{x}\) 在 \(X\) 的一个邻域 \(V\subset X\), 使得 \(V\) 与 \(\mathbb{R}^k\) 的一个开集微分同胚, 则称 \(X\) 是 \(k\) 维光滑流形. 若光滑流形 \(X\) 的子集 \(Y\) 也是光滑流形, 就说 \(Y\) 是 \(X\) 的子流形.
\end{definition}

常见的光滑流形:
\begin{itemize}
  \item \(\mathbb{R}^n\) 是 \(n\) 维光滑流形;
  \item \((0,1)\times\mathbb{R}^n\) 是 \(n+1\) 维光滑流形;
  \item \(\mathbb{R}^n\) 中的单位开球 \(B(1)=\lbrace x\in\mathbb{R}^n: \left\lVert x\right\rVert < 1\rbrace\) 和单位球面 \(S(1)=\lbrace x\in\mathbb{R}^n: \left\lVert x\right\rVert =1\rbrace\) 分别是 \(n\) 维光滑流形和 \(n-1\) 维光滑流形.
\end{itemize}

流形之间光滑映射的任一正则值的逆象是一个光滑流形.

\begin{theorem}[逆象定理]
  设 \(X\) 和 \(Y\) 分别是 \(k\) 维的和 \(l\) 维的光滑流形, \(k>l\), \(f:X\to Y\) 是光滑映射, 如果 \(y\in Y\) 是映射 \(f\) 的正则值, 则 \(f^{-1}(y)\) 或者是空集, 或者是 \(X\) 中的 \(k-l\) 维子流形.
\end{theorem}

逆象定理表明, 正则值的逆象有很好的几何结构. 那么 \(Y\) 中有多少点是光滑映射 \(f:X\to Y\) 的正则值?

\begin{theorem}[Sard 定理]
  设 \(X\) 和 \(Y\) 是光滑流形, \(f:X\to Y\) 是光滑映射. 记 \(D\) 是 \(f\) 的临界点集, 则 \(f\) 的临界点集 \(f(D)\subset Y\) 在 \(Y\) 中的测度为零.
\end{theorem}

\subsection{连续同伦算法}

\begin{definition}
  设 \(X\) 与 \(Y\) 是拓扑空间, \(f_0,f_1:X\to Y\) 是连续映像. 记 \(I=[0,1]\). 若存在连续映像 \(H:X\times I\to Y\), 使得对一切 \(x\in X\) 成立 \(H(x,0)=f_0(x)\) 与 \(H(x,1)=f_1(x)\), 则称 \(f_0\) 同伦于 \(f_1\), 记为 \(f_0\simeq f_1:X\to Y\). 称映像 \(H\) 为从 \(f_0\) 到 \(f_1\) 的\textbf{同伦}或\textbf{伦移}, 记为 \(H:f_0\simeq f_1\) 或 \(f_0 \overset{H}{\simeq} f_1\). 若一个映像 \(f:X\to Y\) 同伦于常值映像, 就说 \(f\) 是一个\textbf{零伦}, 记为 \(f\simeq 0\).
\end{definition}

\begin{theorem}
  记 \(C(X,Y)\) 是 \(X\) 到 \(Y\) 的一切连续映像之集合, 则同伦关系在 \(C(X,Y)\) 中是一种等价关系.
\end{theorem}

根据这一定理, 从 \(X\) 到 \(Y\) 的连续映像集合 \(C(X,Y)\) 按同伦关系可分成若干互不相交的等价类, 其中每一类成为一个\textbf{同伦类}.

常见同伦:
\begin{description}
  \item[线性同伦] \(H(x,t)=tg(x)+(1-t)f(x)\);
  \item[Newton 同伦] \(H(x,t)=\);
\end{description}

\subsubsection{Householder 变换}

\begin{definition}
  对任意单位向量 \(\boldsymbol{u}\in\mathbb{R}^n\), 矩阵
  \begin{equation*}
    Q=I-2\boldsymbol{u}\boldsymbol{u}^\mathrm{T}
  \end{equation*}
  称为关于 \(\boldsymbol{u}\) 的 Householder 变换矩阵.
\end{definition}

Householder 变换的性质:

\begin{itemize}
  \item \(Q\boldsymbol{x}=\boldsymbol{x},\forall\boldsymbol{x}\in P\), 即超平面 \(P\) 中的任何向量不能反射到其他任何地方.
  \item \(Q\boldsymbol{u}=-\boldsymbol{u}\).
  \item Q 是一个正交矩阵, 向量经反射变换 \(Q\) 后不改变长度, \(\left\lVert Q\boldsymbol{x}\right\rVert_2=\left\lVert \boldsymbol{x}\right\rVert_2,\forall\boldsymbol{x}\in\mathbb{R}^n\).
  \item \(Q^2=I\), 即 \(Q\) 是对合的, 从而 \(Q^\mathrm{T}=Q\).
  \item Q 有一个 \(n-1\) 重特征值 \(1\) 和一个单重特征值 \(-1\), 从而有 \(\det Q=-1\).
\end{itemize}

\subsection{单个零点的同伦算法}

\begin{theorem}[广义 Sard 定理, 参数化 Sard 定理]
  设 \(U\subset\mathbb{R}^m,V\subset\mathbb{R}^q\) 是开集, \(\phi:U\times V\to\mathbb{R}^p\) 是 \(C^r\) 映射, \(r\ge\max\lbrace 0,m-p\rbrace\). 若 \(0\in\mathbb{R}^p\) 是 \(\phi\) 的正则值, 则对几乎所有 \(a\in V\), \(0\) 是映射 \(\phi(\cdot,a):U\to\mathbb{R}^p\) 的正则值.
\end{theorem}

\subsection{基本微分方程}

\begin{theorem}
  设 \(\Omega\subset\mathbb{R}^n\) 是一有界开凸集, \(f\) 是映 \(\bar\Omega\) 到其自身上的连续映像, 则 \(f\) 在 \(\bar\Omega\) 中有不动点 \(\boldsymbol{x}^*\), 即 \(f \left(\boldsymbol{x}^*\right)=\boldsymbol{x}^*\). 如果进一步假定 \(f\in C^2(\Omega)\), 则对几乎每个 \(\boldsymbol{x}_0\in\Omega\), 存在一光滑的简单曲线
  \begin{equation*}
    \left\lbrace (x(s),t(s))\in\bar \Omega\times [0,1]: s\in[0,S)\right\rbrace,
  \end{equation*}
  这里 \(S\) 是一正实数或 \(+\infty\), 使得

  (i) \((x(s),t(s))\in H^{-1}(0)\), 其中
  \begin{equation*}
    H(\boldsymbol{x},t)=\boldsymbol{x}-((1-t)\boldsymbol{x}_0+tf(\boldsymbol{x})),
  \end{equation*}
  且 \((x(0),t(0))=(\boldsymbol{x}_0,0)\);

  (ii) \(\lim\limits_{s\to S} t(s)=1\), 因而 \(\lim\limits_{s\to S} (f(x(s))-x(s))=0\);

  (iii) \((x(s),t(s))\) 满足代数微分方程初值问题:
  \begin{equation}
    \begin{cases}
      H_x\dfrac{\mathrm{d} \boldsymbol{x}}{\mathrm{d} s}+H_t \dfrac{\mathrm{d} t}{\mathrm{d} s}=0, \\
      \left\lVert \dfrac{\mathrm{d} \boldsymbol{x}}{\mathrm{d} s}\right\rVert_2^2+\left(\dfrac{\mathrm{d} t}{\mathrm{d} s}\right)^2=1, \\
      x(0)=\boldsymbol{x}_0, \\
      t(0)=0.
    \end{cases}
    \label{eq:homotopy_basic_differential_equation}
  \end{equation}
\end{theorem}

该定理的第一部分是\textbf{Brouwer 不动点定理}, 即:
\begin{theorem}
  设 \(\Omega\) 为 \(\mathrm{R}^n\) 中的有界闭凸集, 映像 \(F:\Omega\to \Omega\) 连续. 则 \(F\) 在 \(\Omega\) 中必有不动点, 即必有 \(\boldsymbol{x}^*\in \Omega\), 使得 \(F \left(\boldsymbol{x}^*\right)=\boldsymbol{x}^*\).
\end{theorem}

后半部分可以看成较弱的 Brouwer 不动点定理的构造型证明. 附加假设 \(f\in C^2(\Omega)\) 较弱是因为实际应用中非光滑算子并不多.

任何求解初值问题 \eqref{eq:homotopy_basic_differential_equation} 的可靠方法都可以用来求解不动点问题 \(\boldsymbol{x}=f(\boldsymbol{x})\). 同伦延拓法的一个较重要的分支实际上是为了求解 \eqref{eq:homotopy_basic_differential_equation} 而设计的算法.

微分方程 \eqref{eq:homotopy_basic_differential_equation} 确定了 \(\left(\dfrac{\mathrm{d} \boldsymbol{x}}{\mathrm{d} s},\dfrac{\mathrm{d} t}{\mathrm{d} s}\right)\) 的两个解向量集(根据其方向相反而区分), 但在计算上是麻烦的. 下述定理解决了这一麻烦:

\begin{theorem}[基本微分方程]
  设 \(H:\bar \Omega\times[0,1]\to\mathbb{R}^n\) 连续可微, \(\Omega\subset\mathbb{R}^n\) 为开集. 假定

  (i) \(\boldsymbol{0}\) 是 \(H\) 的正则值;

  (ii) \(H(\boldsymbol{x}_0,0)=\boldsymbol{0}\) 且 \(\boldsymbol{x}_0\in \Omega\) 是 \(H(\cdot,0)\) 的一个正则点;

  (iii) \(\left\lbrace (\boldsymbol{x}(s),t(s))\in\bar \Omega\times[0,1]: s\in[0,S)\right\rbrace\subset H^{-1}(\boldsymbol{0})\) 是以弧长为参数的同伦道路, 沐足初始条件 \((\boldsymbol{x}(0),t(0))=(\boldsymbol{x}_0,0)\).

  则同伦道路 \((\boldsymbol{x}(s),t(s))\) 由以下微分方程初值问题所确定:

  \begin{equation*}
    \begin{cases}
      \dfrac{\mathrm{d} x_i}{\mathrm{d} s}=(-1)^iC(\boldsymbol{x},t)\cdot\det(\mathrm{DH}_{-i}(\boldsymbol{x},t)), & i=1,2,\cdots,n, \\
      \dfrac{\mathrm{d} t}{\mathrm{d} s}=(-1)^{n+1}C(\boldsymbol{x},t)\cdot\det(H_{\boldsymbol{x}}(\boldsymbol{x},t)), \\
      (\boldsymbol{x}(0),t(0))=(\boldsymbol{x}_0,0),
    \end{cases}
  \end{equation*}
  这里
  \begin{equation*}
    C(\boldsymbol{x},t)=\frac{(-1)^{n+1}\mathrm{sgn}(\det(H_{\boldsymbol{x}}(\boldsymbol{x}_0,0)))}{\sqrt{\sum\limits_{i=1}^{n+1} \left(\det(\mathrm{DH}_{-i})\right)^2}}
  \end{equation*}
  其中 \(\mathrm{sgn}\) 为符号函数.
\end{theorem}

\subsection{路径跟踪算法}

\begin{enumerate}
  \item \textbf{初始化步}. 确定以下三个量:
    \begin{itemize}
      \item 初始点 \((\boldsymbol{x},t)_0=(\boldsymbol{x}_0,0)\);
      \item 初始步长 \(\delta\);
      \item 容许误差 \(\varepsilon\).
    \end{itemize}
  \item \textbf{预估步}. 计算 \((\boldsymbol{x},t)_0\) 处的切向量 \(\left(\dfrac{\mathrm{d} \boldsymbol{x}}{\mathrm{d} s},\dfrac{\mathrm{d} t}{\mathrm{d} s}\right)\), 并且用 Euler 法计算一步得
    \begin{equation*}
      (\boldsymbol{x},t)_1=(\boldsymbol{x},t)_0+\delta \left(\frac{\mathrm{d} \boldsymbol{x}}{\mathrm{d} s},\frac{\mathrm{d} t}{\mathrm{d} s}\right).
    \end{equation*}
    如果 \((\boldsymbol{x},t)_1\) 中的 \(t\) 分量大于 \(1\), 则调整步长 \(\delta\) 使预估在超平面 \(t=1\) 上.
  \item \textbf{校正步}. 以 \((\boldsymbol{x},t)_1\) 为初值, 用迭代法产生一个序列 \(\left\lbrace(\boldsymbol{x},t)_i\right\rbrace_{i=1}^k\), 使 \((\boldsymbol{x},t)_*=(\boldsymbol{x},t)_k\) 为 \(H^{-1}(\boldsymbol{0})\) 中点的近似且其误差小于 \(\varepsilon\). 如果迭代法不收敛, 则缩小步长转回预估步.
  \item \textbf{调换步}. 根据某判别准则, 若 \((\boldsymbol{x},t)_*\) 已满足要求, 则置 \((\boldsymbol{x},t)_0=(\boldsymbol{x},t)_*\), 并调整步长 \(\delta\), 否则缩小步长转预估步.
  \item \textbf{调换步}. 如果 \((\boldsymbol{x},t)_*\) 的 \(t\) 分量等于 \(1\), 停止, 这时 \((\boldsymbol{x},t)_*\) 的 \(\boldsymbol{x}\) 分量是 \(f(\boldsymbol{x})=\boldsymbol{0}\) 的一个近似解. 若 \((\boldsymbol{x},t)_*\) 的 \(t\) 分量小于 \(1\), 转预估步.
\end{enumerate}

\subsection{切向量的计算方法}
